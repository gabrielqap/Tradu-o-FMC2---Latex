
{\large 2.1 Pares ordenados e produto cartesiano.}
\\
\\


Anteriormente, no estudo de conjuntos, certamente foi estudado conjuntos em que continham apenas uma variável livre.
Nesse estudo, nós extenderemos essa ideia para incluir mais uma variável.
\\
\\
Por exemplo, suponha P(x,y) sendo P uma atribuição com dois valores, um para x e outro para y. Nós não podemos afirmar que essa
atribuição é verdadeira ou falsa até que tenhamos um valor para cada variável.
Logo, se nós queremos um conjunto verdadeiro com os valores que fazem essa atribuição ser verdadeira, então esse conjunto não 
contém apenas valores individuais, mas pares de números.
\\
\\
Nós vamos especificar que se nós queremos um par de valores, logo, nós devemos escrevê-los em parenteses, separados por uma 
vírgula. Por exemplo: Seja D(x,y), que significa "x divide Y". Logo, D(6,18) é verdade desde que 6 | 18. 
Entretanto, note que 18 não divide 6. Logo, o par (18,6) com denifição D é falso.
\\
\\
Ou seja, devemos ficar atentos a destinguir o par (18,6) com o par (6,18). Porque a ordem de valores de um par faz a diferença.
Logo, nós vamos nos referir ao par (a,b) como um PAR ORDENADO, Com primeira coordenada A e a segunda coordenada B.

{\large 2.1 Pares ordenados e produto cartesiano.}
\\
\\


Anteriormente, no estudo de conjuntos, certamente foi estudado conjuntos em que continham apenas uma variável livre.
Nesse estudo, nós extenderemos essa ideia para incluir mais uma variável.
\\
\\
Por exemplo, suponha \textit{P}(\textit{x}, \textit{y}) sendo \textit{P} uma atribuição com dois valores, um para \textit{x} 
e outro para \textit{y}. Nós não podemos afirmar que essa atribuição é verdadeira ou falsa até que tenhamos um valor para cada
variável. Logo, se nós queremos um conjunto verdadeiro com os valores que fazem essa atribuição ser verdadeira, então esse 
conjunto não contém apenas valores individuais, mas pares de números.
\\
\\
Nós vamos especificar que se nós queremos um par de valores, logo, nós devemos escrevê-los em parenteses, separados por uma 
vírgula. Por exemplo: Seja \textit{D}(\textit{x},\textit{y}), que significa "\textit{x} divide \textit{y}". Logo,
\textit{D}(\textit{6},\textit{18}) é verdade desde que 6 | 18. Entretanto, note que 18 não divide 6. Logo, o par (18,6) 
com denifição \textit{D} é falso.
\\
\\
Ou seja, devemos ficar atentos a destinguir o par (18,6) com o par (6,18). Porque a ordem de valores de um par faz a diferença.
Logo, nós vamos nos referir ao par (\textit{a},\textit{b}) como um \textit{par ordenado}, Com \textit{primeira coordenada a e 
a segunda coordenada B}.
\\
\\
Você provavelmente já viu pares ordenados antes, estudando pontos num plano cartesiano. O uso de \textit{x} e \textit{y} para 
identificar pontos no plano funciona assumindo que cada ponto no plano um par ordenado, os quais, \textit{x} e \textit{y} são
coordenadas do ponto. Os pares devem estar ordenados porque, por exemplo, os pontos (2,5) e (5,2) são pontos diferentes no plano.
Nesse caso as coordenadas de pares ordenados são numeros reais, mas pares ordenados podem haver qualquer coisa em suas coordenadas.
\\
Por exemplo, seja \textit{C}(\textit{x}, \textit{y}), que suporta que "\textit{x} tem \textit{y} crianças". Nesse afirmação, a 
variável \textit{x} abrange o conjunto de todas as pessoas e \textit{y} abrange o conjunto de todos números naturais. Logo, os 
únicos pares ordenados que fazem sentido são aqueles em que a primeira coordenada é uma pessoa e a segunda coordenada é um número
natural.
\\

Portanto, o unico par ordenado que faz sentido considerar quando discutimos atribuições de valores para as variáveis \textit{x} e
\textit{y} nessa atribuição são pares no qual a primeira é uma pessoa e a segunda um número natural.
\\
Por exemplo, a atribuição (Príncipe Charles, 2) faz \textit{C}(\textit{x}, \textit{y}) ser verdade, pois o Príncipe Charles de fato 
tem dois filhos, onde a atribuição (Jhonny Carson, 37) faz a declaração ser falsa. Note que a atribuição (2, Príncipe Charles) não 
faz sentido, porque leva a uma declaração ``2 tem Príncipe Charles filhos''. 
\\
\\
No geral, se \textit{P}(\textit{x}, \textit{y}) é uma declaração no qual \textit{x} varia sobre um set \textit{A} e \textit{y} varia
sobre um set \textit{B}, então a única atribuição de valores para \textit{x} e \textit{y} que irá fazer sentido em \textit{P}
(\textit{x}, \textit{y}) vão ser os pares ordenados no qual a primeira coordenada é um elemento de \textit{A} e a segunda é de
\textit{B}. Logo, fazemos a seguinte definição:
\\

\textbf{Definição 2.1.1.} suponha \textit{A} e \textit{B} são sets arbitrários. Então o \textit{produto cartesiano} de
\textit{A} e \textit{B}, denotado de \textit{A} $\times$ \textit{B}, é o set de todo par ordenado no qual a primeira 
coordenada é um elemento de \textit{A} e a segunda é um elemento de \textit{B}. Em outras palavras, 
\textit{A} $\times$ \textit {B} = \{(\textit{a}, \textit{b})|\textit{a} $\in$ \textit{A} e \textit{b} $\in$ \textit{B}\}.
\\

\textbf{Exemplo 2.1.2.}
\\
1. Se \textit{A} = \{red, green\} e \textit{B} = \{2, 3, 5\} então \textit{A} $\times$ \textit{B} = \{(red, 2), (red, 3), (red,5),
 (green, 2), (green, 3), (green, 5)\}.
\\
2. Se \textit{P} = o set de todas as pessoas  entao \textit{P} $\times$ $\mathbb{N}$ = \{(\textit{p}, \textit{n})|\textit{p} é uma 
pessoa e \textit{n} um número natural\} = \{(Príncipe Charles, 0), (Príncipe Charles, 1), (Príncipe Charles, 2),...,
(Johnny Carson, 0), (Johnny Carson, 1),...\}. Esses pares ordenados fazem sentido como atribuição de valores de variáveis livres
\textit{x} e \textit{y} na declaração \textit{C}(\textit{x}, \textit{y}).
\\
3. $\mathbb{R}$ $\times$ $\mathbb{R}$ = \{(\textit{x}, \textit{y})|\textit{x} e \textit{y} são números reais\}. Essas são as 
coordenadas de todos os pontos do plano. Por razôes óbvias, este conjunto geralmente é escrito $\mathbb{R}^2$.
\\
\\

\textbf{Teorema 2.1.3.} \textit{Suponha A, B, C, e D são conjuntos.}
\\
1. \textit{A} $\times$ (\textit{B} $\cap$ \textit{C}) = (\textit{A} $\times$ \textit{B}) $\cap$ (\textit{A} $\times$ \textit{C}).
\\
2. \textit{A} $\times$ (\textit{B} $\cup$ \textit{C}) = (\textit{A} $\times$ \textit{B}) $\cup$ (\textit{A} $\times$ \textit{C}).
\\
3. (\textit{A} $\times$ \textit{B}) $\cap$ (\textit{C} $\times$ \textit{D}) = (\textit{A} $\cap$ \textit{C}) $\times$ (\textit{B}
$\cap$ \textit{D}).
\\
4. (\textit{A} $\times$ \textit{B}) $\cup$ (\textit{C} $\times$ \textit{D}) = (\textit{A} $\cup$ \textit{C}) $\times$ (\textit{B}
$\cup$ \textit{D}).
\\
5. \textit{A} $\times$ $\emptyset$ = $\emptyset$ $\times$ \textit{A} = $\emptyset$.  
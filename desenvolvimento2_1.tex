
{\large 2.1 Pares ordenados e produto cartesiano.}
\\
\\


Anteriormente, no estudo de conjuntos, certamente foi estudado conjuntos em que continham apenas uma variável livre.
Nesse estudo, nós extenderemos essa ideia para incluir mais uma variável.
\\
\\
Por exemplo, suponha P(x,y) sendo P uma atribuição com dois valores, um para x e outro para y. Nós não podemos afirmar que essa
atribuição é verdadeira ou falsa até que tenhamos um valor para cada variável.
Logo, se nós queremos um conjunto verdadeiro com os valores que fazem essa atribuição ser verdadeira, então esse conjunto não 
contém apenas valores individuais, mas pares de números.
\\
\\
Nós vamos especificar que se nós queremos um par de valores, logo, nós devemos escrevê-los em parenteses, separados por uma 
vírgula. Por exemplo: Seja D(x,y), que significa "x divide Y". Logo, D(6,18) é verdade desde que 6 | 18. 
Entretanto, note que 18 não divide 6. Logo, o par (18,6) com denifição D é falso.
\\
\\
Ou seja, devemos ficar atentos a destinguir o par (18,6) com o par (6,18). Porque a ordem de valores de um par faz a diferença.
Logo, nós vamos nos referir ao par (a,b) como um PAR ORDENADO, Com primeira coordenada A e a segunda coordenada B.
\\
\\
Você provavelmente já viu pares ordenados antes, estudando pontos num plano cartesiano. O uso de x e y para identificar pontos 
no plano funciona assumindo que cada ponto no plano um par ordenado, os quais, x e y são coordenadas do ponto. Os pares devem 
estar ordenados porque, por exemplo, os pontos (2,5) e (5,2) são pontos diferentes no plano. Nesse caso as coordenadas de pares
ordenados são numeros reais, mas pares ordenados podem haver qualquer coisa em suas coordenadas.
\\
Por exemplo, seja C(x,y), que suporta que "x tem y crianças". Nesse afirmação, a variável x abrange o conjunto de todas as 
pessoas e y abrange o conjunto de todos números naturais. Logo, os únicos pares ordenados que fazem sentido são aqueles em
que a primeira coordenada é uma pessoa e a segunda coordenada é um número natural.
{\textbf{\large 2.2 Relações.}}
\\
\\

Suponha \textit{P}(\textit{x}, \textit{y}) é uma declaração com duas varáveis livres \textit{x} e \textit{y}. Geralmente essa
declaração pode ser expressada como uma realação de entre \textit{x} e \textit{y}. O verdadeiro conjunto da declaração
\textit{P}(\textit{x}, \textit{y}) é um conjunto de pares ordenados quando essa relação é válida. De fato, geralmente é útil 
pensar que qualquer conjunto de pares ordenados como um registro quando alguma relação é verdadeira. Esta é a motivação por tras
da definição seguinte.
\\
\\

\textbf{Definição 2.2.1.} Suponha \textit{A} e \textit{B} são conjuntos. Então um conjunto \textit{R} $\subseteq$ \textit{A} 
$\times$ \textit{B} é chamada uma \textit{relação de A para B}.
\\
Se \textit{x} está em \textit{A} e \textit{y} está em \textit{B}, então claramente o conjunto de qualquer declaração \textit{P}
(\textit{x}, \textit{y}) será uma relação de \textit{A} para \textit{B}. No entanto, note que a definição 2.2.1 não requer que 
o conjunto de pares ordenados seja definido como o conjunto de alguma declaração para o conjunto ser uma relação. Apesar de pensar
sobre os conjuntos era a motivação para essa definição, a definição diz nada explícito sobre os conjuntos. De acordo com a definição,
qualquer subconjunto de \textit{A} $\times$ \textit{B} é chamado uma relação de \textit{A} para \textit{B}.
\\
\\

\textbf{Exemplo 4.2.2.} Aqui estão alguns exemplos de relações de um conjunto para outro.
\\
1. Seja \textit{A} = \{1, 2, 3\}, \textit{B} = \{3, 4, 5\}, e \textit{R} = \{(1, 3), (1,5), (3, 3)\}. Então \textit{R} $\subseteq$ 
\textit{A} $\times$ \textit{B}, então \textit{R} é uma relação de \textit{A} para \textit{B}.
\\
2. Seja \textit{G} = \{(\textit{x}, \textit{y}) $\in$ $\mathbb{R}$ $\times$ $\mathbb{R}$|\textit{x} > \textit{y}\}. Então \textit{G}
é uma relação de $\mathbb{R}$ para $\mathbb{R}$.
\\
3. Seja \textit{A} = \{1, 2\} e \textit{B} = $\mathcal P(A)$ = \{$\emptyset$, \{1\}, \{2\}, \{1, 2\}\}. Seja \textit{E} = 
\{(\textit{x},\textit{y}) $\in$ \textit{A} $\times$ \textit{B}|\textit{x} $\in$ \textit{y}\}. Então \textit{E} é uma relação de 
\textit{A} para \textit{B}. Neste caso, \textit{E} = \{(1, \{1\}), (1, \{1, 2\}), (2, \{2\}), (2, \{1, 2\})\}.
\\
Para os próximos três exemplos, seja \textit{S} o conjunto de todos os estudantes da sua escola, \textit{R} o conjunto de todos os
dormitórios, \textit{P} o conjunto de todos os professores e \textit{C} o conjunto de todos os cursos.
\\
4. Seja \textit{L} = \{(\textit{s}, \textit{r}) $\in$ \textit{S} $\times$ \textit{R} | o estudante \textit{s} mora no dormitório 
\textit{r}\}. Então \textit{L} é uma relação de \textit{S} para \textit{R}.
\\
5. Seja \textit{E} = \{(\textit{s}, \textit{c}) $\in$ \textit{S} $\times$ \textit{C} | o estudante \textit{s} está matriculado no 
curso \textit{c}\}. Então \textit{E} é uma relação de \textit{S} para \textit{C}.
\\
6. Seja \textit{T} = \{(\textit{c}, \textit{p}) $\in$ \textit{C} $\times$ \textit{P} | o curso \textit{c} é lecionado pelo professor
\textit{p}\}. Então \textit{T} é uma relação de \textit{C} para \textit{P}.
\\
\\
Aqui temos definições para vários conceitos novos envolvendo relações. Logo iremos dar exemplos ilustrados desses conceitos,
mas primeiro veja se você consegue entender os conceitos baseado nas suas definições.
\\
\\
\textbf{Definição 2.2.3.} Suponha \textit{R} é uma relação de \textit{A} para \textit{B}. Então o dominio de \textit{R} é o
conjunto

\begin{center}
Dom(\textit{R}) = \{\textit{a} $\in$ \textit{A} | $\exists$\textit{b} $\in$ \textit{B}((\textit{a}, \textit{b}) $\in$ \textit{R})\}. 
\end{center}
A imagem de \textit{R} é um conjunto

\begin{center}
Ran(\textit{R}) = \{\textit{b} $\in$ \textit{B} | $\exists$\textit{a} $\in$ \textit{A}((\textit{a}, \textit{b}) $\in$ \textit{R})\}. 
\end{center}
O inverso de \textit{R} é a relação $\textit{R}^{-1}$ de \textit{B} para \textit{A} definido como:

\begin{center}
$\textit{R}^{-1}$ = \{(\textit{b}, \textit{a}) $\in$ \textit{B} $\times$ \textit{A} | (\textit{a}, \textit{b}) $\in$ \textit{R}\}. 
\end{center}

Finalmente, suponha \textit{R} uma relação de \textit{A} para \textit{B} e \textit{S} é uma relação de \textit{B} para \textit{C}.
Então a composição de \textit{S} e \textit{R} é uma relação \textit{S} $\circ$ \textit{R} de \textit{A} para \textit{C} definido 
como:

\begin{center}
\textit{S} $\circ$ \textit{R} = \{(\textit{a}, \textit{c}) $\in$ \textit{A} $\times$ \textit{C} | $\exists$\textit{b} $\in$
\textit{B}((\textit{a}, \textit{b}) $\in$ \textit{R} e (\textit{b}, \textit{c}) $\in$ \textit{S})\}. 
\end{center}

{\textbf{\large 4.2 Relações.}}
\\
\\

Suponha \textit{P}(\textit{x}, \textit{y}) é uma declaração com duas varáveis livres \textit{x} e \textit{y}. Geralmente essa
declaração pode ser expressada como uma realação de entre \textit{x} e \textit{y}. O verdadeiro conjunto da declaração
\textit{P}(\textit{x}, \textit{y}) é um conjunto de pares ordenados quando essa relação é válida. De fato, geralmente é útil 
pensar que qualquer conjunto de pares ordenados como um registro quando alguma relação é verdadeira. Esta é a motivação por tras
da definição seguinte.
\\
\\

\textbf{Definição 4.2.1.} Suponha \textit{A} e \textit{B} são conjuntos. Então um conjunto \textit{R} $\subseteq$ \textit{A} 
$\times$ \textit{B} é chamada uma \textit{relação de A para B}.
\\
Se \textit{x} está em \textit{A} e \textit{y} está em \textit{B}, então claramente o conjunto de qualquer declaração \textit{P}
(\textit{x}, \textit{y}) será uma relação de \textit{A} para \textit{B}. No entanto, note que a definição 4.2.1 não requer que 
o conjunto de pares ordenados seja definido como o conjunto de alguma declaração para o conjunto ser uma relação. Apesar de pensar
sobre os conjuntos era a motivação para essa definição, a definição diz nada explícito sobre os conjuntos. De acordo com a definição,
qualquer subconjunto de \textit{A} $\times$ \textit{B} é chamado uma relação de \textit{A} para \textit{B}.
\\
\\

\textbf{Exemplo 4.2.2.} Aqui estão alguns exemplos de relações de um conjunto para outro.
\\
1. Seja \textit{A} = \{1, 2, 3\}, \textit{B} = \{3, 4, 5\}, e \textit{R} = \{(1, 3), (1,5), (3, 3)\}. Então \textit{R} $\subseteq$ 
\textit{A} $\times$ \textit{B}, então \textit{R} é uma relação de \textit{A} para \textit{B}.
\\
2. Seja \textit{G} = \{(\textit{x}, \textit{y}) $\in$ $\mathbb{R}$ $\times$ $\mathbb{R}$|\textit{x} > \textit{y}\}. Então \textit{G}
é uma relação de $\mathbb{R}$ para $\mathbb{R}$.
\\
3. Seja \textit{A} = \{1, 2\} e \textit{B} = $\mathcal P(A)$ = \{$\emptyset$, \{1\}, \{2\}, \{1, 2\}\}. Seja \textit{E} = 
\{(\textit{x},\textit{y}) $\in$ \textit{A} $\times$ \textit{B}|\textit{x} $\in$ \textit{y}\}. Então \textit{E} é uma relação de 
\textit{A} para \textit{B}. Neste caso, \textit{E} = \{(1, \{1\}), (1, \{1, 2\}), (2, \{2\}), (2, \{1, 2\})\}.
\\	
Para os próximos três exemplos, seja \textit{S} o conjunto de todos os estudantes da sua escola, \textit{R} o conjunto de todos os
dormitórios, \textit{P} o conjunto de todos os professores e \textit{C} o conjunto de todos os cursos.
\\
4. Seja \textit{L} = \{(\textit{s}, \textit{r}) $\in$ \textit{S} $\times$ \textit{R} | o estudante \textit{s} mora no dormitório 
\textit{r}\}. Então \textit{L} é uma relação de \textit{S} para \textit{R}.
\\
5. Seja \textit{E} = \{(\textit{s}, \textit{c}) $\in$ \textit{S} $\times$ \textit{C} | o estudante \textit{s} está matriculado no 
curso \textit{c}\}. Então \textit{E} é uma relação de \textit{S} para \textit{C}.
\\
6. Seja \textit{T} = \{(\textit{c}, \textit{p}) $\in$ \textit{C} $\times$ \textit{P} | o curso \textit{c} é lecionado pelo professor
\textit{p}\}. Então \textit{T} é uma relação de \textit{C} para \textit{P}.
\\
\\
Aqui temos definições para vários conceitos novos envolvendo relações. Logo iremos dar exemplos ilustrados desses conceitos,
mas primeiro veja se você consegue entender os conceitos baseado nas suas definições.
\\
\\
\textbf{Definição 4.2.3.} Suponha \textit{R} é uma relação de \textit{A} para \textit{B}. Então o dominio de \textit{R} é o
conjunto

\begin{center}
Dom(\textit{R}) = \{\textit{a} $\in$ \textit{A} | $\exists$\textit{b} $\in$ \textit{B}((\textit{a}, \textit{b}) $\in$ \textit{R})\}. 
\end{center}
A imagem de \textit{R} é um conjunto

\begin{center}
Ran(\textit{R}) = \{\textit{b} $\in$ \textit{B} | $\exists$\textit{a} $\in$ \textit{A}((\textit{a}, \textit{b}) $\in$ \textit{R})\}. 
\end{center}
O inverso de \textit{R} é a relação $\textit{R}^{-1}$ de \textit{B} para \textit{A} definido como:

\begin{center}
$\textit{R}^{-1}$ = \{(\textit{b}, \textit{a}) $\in$ \textit{B} $\times$ \textit{A} | (\textit{a}, \textit{b}) $\in$ \textit{R}\}. 
\end{center}

Finalmente, suponha \textit{R} uma relação de \textit{A} para \textit{B} e \textit{S} é uma relação de \textit{B} para \textit{C}.
Então a composição de \textit{S} e \textit{R} é uma relação \textit{S} $\circ$ \textit{R} de \textit{A} para \textit{C} definido 
como:

\begin{center}
\textit{S} $\circ$ \textit{R} = \{(\textit{a}, \textit{c}) $\in$ \textit{A} $\times$ \textit{C} | $\exists$\textit{b} $\in$
\textit{B}((\textit{a}, \textit{b}) $\in$ \textit{R} e (\textit{b}, \textit{c}) $\in$ \textit{S})\}. 
\end{center}

Note que nos assumimos que a segunda coordenada de pares em \textit{R} e a primeira coordenada de pares em \textit{S} vem do mesmo conjunto, \textit{B}. Se esses conjuntos não são o mesmo, a composição \textit{S} $\circ$ \textit{R} seria indefinida.
\\

De acordo com definição 4.2.3, o domínio da relação de \textit{A} para \textit{B} é o conjunto contendo todas as primeiras coordenadas dos pares ordenados da relação. Isso vai geralmente é o subconjunto de \textit{A}, mas não todo o \textit{A}. Por exemplo, considere a relação \textit{L} da parte 4 do Exemplo 4.2.2, No qual os pares mostram o estudante e o dormitório em que ele mora. O domínio de \textit{L} irá conter todos os estudantes que aparecem na primeira coordenada em algum par ordenado em \textit{L} - em outras palavras, todos os estudantes que moram em algum dormitório - mas não irá conter, por exemplo, os estudantes que moram em apartamento fora do campus. Extraindo mais cuidadosamente da definição da declaração, nós temos

\begin{center}
Dom(\textit{L}) = \{\textit{s} $\in$ \textit{S} | $\exists$\textit{r} $\in$ \textit{R}((\textit{s}, \textit{r}) $\in$ \textit{L})\}
\\
= \{\textit{s} $\in$ \textit{S} | $\exists$\textit{r} $\in$ \textit{R}(o estudante \textit{s} mora no dormitório \textit{r})\}
\\
= \{\textit{s} $\in$ \textit{S} | o estudante \textit{s} mora em algum dormitório\}.
\end{center}

Semelhantemente, a imagem da relação é o conjunto contendo todas as segundas coordenadas dos pares ordenados. Por exemplo, a imagem da relação \textit{L} seria o conjunto de todos os dormitórios no qual mora algum estudante. Qualquer dormitório não está ocupado não seria uma imagem de \textit{L}.
\\
O inverso da relação contem exatamente o mesmo par ordenado da relação original, mas com a ordem das coordenadas de cada par invertido. Desse modo, no caso da relação \textit{L}, se Joe Smith mora no quarto 213 Davis Hall, então (Joe Smith, 213 Davis Hall) $\in$ \textit{L} e (213 David Hall, Joe Smith) $\in$ $\textit{L}^{-1}$. No geral, para qualquer estudante \textit{s} e o dormitório \textit{r}, nós temos (\textit{r}, \textit{s}) $\in$ $\textit{L}^{-1}$ sse (\textit{s}, \textit{r}) $\in$ \textit{L}. Para outro exemplo, considere a relação \textit{G} da parte 2 do Exemplo 2.2.2. Contém todos os pares ordenados de números reais (\textit{x}, \textit{y}) no qual \textit{x} é maior do que \textit{y}. Nós podemos chamar de relação "maior do que". Seu inverso é 

\begin{center}
$\textit{G}^{-1}$ = \{(\textit{x}, \textit{y}) $\in$ $\mathbb{R}$ $\times$ $\mathbb{R}$ | (\textit{y}, \textit{x}) $\in$ \textit{G}\}
\\
= \{(\textit{x}, \textit{y}) $\in$ $\mathbb{R}$ $\times$ $\mathbb{R}$ | \textit{y} > \textit{x}\}
\\
= \{(\textit{x}, \textit{y}) $\in$ $\mathbb{R}$ $\times$ $\mathbb{R}$ | \textit{y} < \textit{x}\}
\end{center}

Em outras palavras, o inverso da relação "maior que" é a relação "menor que"!
\\
\\

O conceito mais difícil introduzido na definição 4.2.3 é o conceito de composição de duas relações. Um exemplo desse conceito, considere as relações \textit{E} e \textit{T} das partes 5 e 6 do Exemplo 4.2.2. Relembrando que \textit{E} é uma relação do conjunto \textit{S} de todos os estudantes para o conjunto \textit{C} de todos os cursos, e \textit{T} é a relação de \textit{C} para o conjunto \textit{P} de todos os professores. De acordo com a definição 2.2.3, a composição \textit{T} $\circ$ \textit{E} será a relação de \textit{S} para \textit{P} definida abaixo:

\begin{center}
\textit{T} $\circ$ \textit{E} = \{(\textit{s}, \textit{p}) $\in$ \textit{S} $\times$ \textit{P} | $\exists$\textit{c} $\in$ \textit{C} ((\textit{s}, \textit{c}) $\in$ \textit{E} e (\textit{c}, \textit{p}) $\in$ \textit{T})\}
\\
= \{(\textit{s}, \textit{p}) $\in$ \textit{S} $\times$ \textit{P} | $\exists$\textit{c} $\in$ \textit{C}(o estudante \textit{s} está matriculado no curso \textit{c} e o curso \textit{c} é lecionado pelo professor \textit{p})\}
\\
= \{(\textit{s}, \textit{p}) $\in$ \textit{S} $\times$ \textit{P} | o estudante \textit{s} está matriculado em algum curso lecionado pelo professor \textit{p}\}.
\end{center}

Portanto, se Joe Smith está matriculado em Biologia 12 e Biologia 12 é lecionada pelo professor Evans, então (Joe Smith, Biology 12) $\in$ \textit{E} e (Biologia 12, Professor Evans) $\in$ \textit{T}, Logo (Joe Smith, Professor Evans) $\in$ \textit{T} $\circ$ \textit{E}. No geral, se \textit{s} é algum estudante em particular e \textit{p} algum professor em particular, então (\textit{s}, \textit{p}) $\in$ \textit{T} $\circ$ \textit{E} sse existe algum curso c tal que (\textit{s}, \textit{c}) $\in$ \textit{E} e (\textit{c}, \textit{p}) $\in$ \textit{T}. Esta notação pode parecer meio retrógrado analisando pela primeira vez. Se (\textit{s}, \textit{c}) $\in$ \textit{E} e (\textit{c}, \textit{p}) $\in$ \textit{T}, então você pode ficar tentado a escrever (\textit{s}, \textit{p}) $\in$ \textit{E} $\circ$ \textit{T}, mas de acordo com a nossa definição, a notação adequada é (\textit{s}, \textit{p}) $\in$ \textit{T} $\circ$ \textit{E}. De fato, \textit{E} $\circ$ \textit{T} é indefinida, porque a segunda coordenada de pares ordenados em \textit{T} e a primeira coordenada de pares em \textit{E} não vem do mesmo conjunto.   
\\

\textbf{Exemplo 4.2.4.} Seja \textit{S}, \textit{R}, \textit{C}, e \textit{P} os conjuntos de estudantes, dormitórios, cursos e professores na sua escola, como antes, e seja \textit{L}, \textit{E}, e \textit{T} as relações definidas nas partes 4-6 do Exemplo 4.2.2. Descreva as relações seguintes.
\\
1. $\textit{E}^{-1}$
\\
2. \textit{E} $\circ$ $\textit{L}^{-1}$.
\\
3. $\textit{E}^{-1}$ $\circ$ \textit{E}.
\\
4. \textit{E} $\circ$ $\textit{E}^{-1}$.
\\
5. \textit{T} $\circ$ (\textit{E} $\circ$ $\textit{L}^{-1}$).
\\
6. (\textit{T} $\circ$ \textit{E}) $\circ$ $\textit{L}^{-1}$
\\

\textit{Soluções}
\\
\\
1. $\textit{E}^{-1}$ = \{(\textit{c}, \textit{s}) $\in$ \textit{C} $\times$ \textit{S} | (\textit{s}, \textit{c}) $\in$ \textit{E}\} = \{(\textit{c}, \textit{s}) $\in$ \textit{C} $\times$ \textit{S} | o estudante \textit{s} está matriculado no curso \textit{c}\}. Por exemplo, se Joe Smith está matriculado em Biologia 12, então (Joe Smith, Biology 12) $\in$ \textit{E} e (BIologia 12, Joe Smith) $\in$ $\textit{E}^{-1}$.
\\
2. Porque $\textit{L}^{-1}$ é uma relação de \textit{R} para \textit{S} e \textit{E} é uma relação de \textit{S} para \textit{C}, \textit{E} $\circ$ $\textit{L}^{-1}$ será a relação de \textit{R} para \textit{C} definida abaixo.
\begin{center}
\textit{E} $\circ$ $\textit{L}^{-1}$ = \{(\textit{r}, \textit{c}) $\in$ \textit{R} $\times$ \textit{C} | $\exists$\textit{s} $\in$ \textit{S}((\textit{r}, \textit{s}) $\in$ $\textit{L}^{-1}$ e (\textit{s}, \textit{c}) $\in$ \textit{E})\}
\\
= \{(\textit{r}, \textit{c}) $\in$ \textit{R} $\times$ \textit{C} | $\exists$\textit{s} $\in$ \textit{S}((\textit{r}, \textit{s}) $\in$ \textit{L} e (\textit{s}, \textit{c}) $\in$ \textit{E})\}
\\
= \{(\textit{r}, \textit{c}) $\in$ \textit{R} $\times$ \textit{C} | $\exists$\textit{s} $\in$ \textit{S}(o estudante \textit{s} mora em um dormitório \textit{r} e está matriculado no curso \textit{c})\}
\\
= \{(\textit{r}, \textit{c}) $\in$ \textit{R} $\times$ \textit{C} | algum estudante que mora no dormitório \textit{r} está matriculado no curso \textit{c}\}.
\\
Retornando ao estudante Joe Smith, que está matriculado em Biologia 12 e mora no dormitório 213 Davis Hall, nós temos (213 Davis Hall, Joe Smith) $\in$ $\textit{L}^{-1}$ e (Joe Smith, Biologia 12) $\in$ \textit{E}, e portanto (213 Davis Hall, Biologia 12) $\in$ \textit{E} $\circ$ $\textit{L}^{-1}$.
\\
3. Porque \textit{E} é uma relação de \textit{S} para \textit{C} e $\textit{E}^{-1}$ é uma relação de \textit{C} para \textit{S}, $\textit{E}^{-1}$ $\circ$ \textit{E} é uma relação de \textit{S} para \textit{S} definida abaixo.
\\
$\textit{E}^{-1}$ $\circ$ \textit{E} = \{(\textit{s}, \textit{t}) $\in$ \textit{S} $\times$ \textit{S} | $\exists$\textit{c} $\in$ \textit{C}((\textit{s}, \textit{c}) $\in$ \textit{E} e (\textit{c}, \textit{t}) $\in$ $\textit{E}^{-1}$)\}
\\
= \{(\textit{s}, \textit{t}) $\in$ \textit{S} $\times$ \textit{S} | $\exists$\textit{c} $\in$ \textit{C}(o estudante \textit{s} está matriculado no curso \textit{c}, e o estudante \textit{t} também)\}
\\
= \{(\textit{s}, \textit{t}) $\in$ \textit{S} $\times$ \textit{S} | existe algum curso  que os estudantes \textit{s} e \textit{t} estão matriculados\}.
\end{center}
(Note que um elemento arbitrário de \textit{S} $\times$ \textit{S} é escrito (\textit{s}, \textit{t}), não (\textit{s}, \textit{s}), porque nós não assumimos que as duas coordenadas são iguais.)
\\
4. Esse não é o mesmo do exemplo anterior! Porque $\textit{E}^{-1}$ é uma relação de \textit{C} para \textit{S} e \textit{E} é uma relação de \textit{S} para \textit{C}, \textit{E} $\circ$ $\textit{E}^{-1}$ é uma relação de \textit{C} para \textit{C}. Definida abaixo.
\begin{center}
\textit{E} $\circ$ $\textit{E}^{-1}$ = \{(\textit{c}, \textit{d}) $\in$ \textit{C} $\times$ \textit{C} |$\exists$\textit{s} $\in$ \textit{S}((\textit{c}, \textit{s}) $\in$ $\textit{E}^{-1}$ e (\textit{s}, \textit{d}) $\in$ \textit{E})\}
\\
= \{(\textit{c}, \textit{d}) $\in$ \textit{C} $\times$ \textit{C} | $\exists$\textit{s} $\in$ \textit{S}(o estudante \textit{s} está matriculado no curso \textit{c}, e ele também está matriculado no curso \textit{d})\}

= \{(\textit{c}, \textit{d}) $\in$ \textit{C} $\times$ \textit{C} | existe algum estudante que está matriculado nos cursos \textit{c} e \textit{d}\}.
\end{center}

5. Nós vimos na parte 2 que \textit{E} $\circ$ $\textit{L}^{-1}$ é a relação de \textit{R} para \textit{C}, e \textit{T} é a relação de \textit{C} para \textit{P}, então \textit{T} $\circ$ (\textit{E} $\circ$ $\textit{L}^{-1}$) é a relação de \textit{R} para \textit{P} definida abaixo.
\begin{center}
\textit{T} $\circ$ (\textit{E} $\circ$ $\textit{L}^{-1}$) = \{(\textit{r}, \textit{p}) $\in$ \textit{R} $\times$ \textit{P} |$\exists$\textit{c} $\in$ \textit{C}((\textit{r}, \textit{c}) $\in$ \textit{E} $\circ$ $\textit{L}^{-1}$ e (\textit{c}, \textit{p}) $\in$ \textit{T})\}
\\
= \{(\textit{r}, \textit{p}) $\in$ \textit{R} $\times$ \textit{P} |$\exists$\textit{c} $\in$ \textit{C}(algum estudante que mora no dormitório \textit{r} está matriculado no curso \textit{c}, e \textit{e} é lecionada pelo professor \textit{p})\}
\\
= \{(\textit{r}, \textit{p}) $\in$ \textit{R} $\times$ \textit{P} | algum estudante que mora no dormitório \textit{r} está matriculado em algum curso lecionado pelo professor \textit{p}\}.
\end{center}

6. 
\begin{center}
(\textit{T} $\circ$ \textit{E}) $\circ$ $\textit{L}^{-1}$ = \{(\textit{r}, \textit{p}) $\in$ \textit{R} $\times$ \textit{P} |$\exists$\textit{s} $\in$ \textit{S}((\textit{r}, \textit{s}) $\in$ $\textit{L}^{-1}$ e (\textit{s}, \textit{p}) $\in$ \textit{T} $\circ$ \textit{E})\}
\\
= \{(\textit{r}, \textit{p}) $\in$ \textit{R} $\times$ \textit{P} |$\exists$\textit{s} $\in$ \textit{S}(o estudante \textit{s} mora no dormitório \textit{r}, e está matriculado em um curso lecionado pelo professor \textit{p})\}
\\
= \{(\textit{r}, \textit{p}) $\in$ \textit{R} $\times$ \textit{P} | algum estudante que mora no dormitório \textit{r} está matriculado no curso lecionado pelo professor \textit{p}\}.
\end{center}

Note que nossas respostas para as partes 3 e 4 do exemplo 4.2.4 foram diferentes. Logo a composição das relações não é comutativa. No entanto, nossas respostas para as partes 5 e 6 são as mesmas. Isto é uma coincidência, ou no geral é verdadeiro que a composição de relações é associativa? Geralmente, olhando para os exemplos de uma nova concepção irão surgir regras que talvez sejam aplicáveis. Apesa de um contra exemplo é o bastante para mostrar que a regra é incorreta, nós devemos nunca aceitar uma regra como correta sem uma prova. O próximo teorema resume algumas propriedades basicas dos novos conceitos que nós introduzimos.
\\

\textbf{Teorema 4.2.5.} Suponha \textit{R} é uma relação de \textit{A} para \textit{B}, \textit{S} é uma relação de \textit{B} para \textit{C}, e \textit{T} é uma relação de \textit{C} para \textit{D}. Então:
\\
\\
1. $(\textit{R}^{-1})^{-1}$ = \textit{R}.
\\
2. Dom($\textit{R}^{-1}$) = Ran(\textit{R}).
\\
3. Ran($\textit{R}^{-1}$) = Dom(\textit{R}).
\\
4. \textit{T} $\circ$ (\textit{S} $\circ$ \textit{R}) = (\textit{T} $\circ$ \textit{S}) $\circ$ \textit{R}.
\\
5. $(\textit{S} \circ \textit{R})^{-1}$ = $\textit{R}^{-1} \circ \textit{S}^{-1}$.
\\

Prova. Nós iremos provar 1,2 e metade do 4 e deixar o resto como exercício.
\\
1. Primeiramente, note que $\textit{R}^{-1}$ é uma relação de \textit{B} para \textit{A}, então $(\textit{R}^{-1})^{-1}$ é uma relação de \textit{A} para \textit{B}, assim como \textit{R}. Para ver que $(\textit{R}^{-1})^{-1} = \textit{R}$, seja (\textit{a}, \textit{b}) um par ordenado arbitrário em $\textit{A} \times \textit{B}$. Então
\begin{center}
$(\textit{a}, \textit{b}) \in (\textit{R}^{-1})^{-1} sse (\textit{b}, \textit{a}) \in \textit{R}^{-1} sse (\textit{a}, \textit{b}) \in \textit{R}$.
\end{center}
2. Primeiro note que Dom($\textit{R}^{-1}$) e Ran(\textit{R}) são ambos subconjuntos de \textit{B}. Agora seja \textit{b} um elemento arbitrário de \textit{B}. Então
\begin{center}
$\textit{b} \in Dom(\textit{R}^{-1})$ sse $\exists\textit{a} \in \textit{A}((\textit{b}, \textit{a}) \in \textit{R}^{-1})$ sse $\exists\textit{a} \in \textit{A}((\textit{a}, \textit{b}) \in \textit{R})$ sse $\textit{b} \in Ran(\textit{R})$.
\end{center}

4. Claramente $\textit{T} \circ (\textit{S} \circ \textit{R})$  e  $(\textit{T} \circ \textit{S}) \circ \textit{R}$  são ambas relações de \textit{A} para \textit{D}. Seja (\textit{a}, \textit{d}) um elemento arbitrário de $\textit{A} \times \textit{D}$.
\\

Primeiro, suponha $(\textit{a}, \textit{d}) \in \textit{T} \circ (\textit{S} \circ \textit{R})$. Pela definição de composição, isto significa que nós podemos escolher algum $\textit{c} \in \textit{C}$ tal que $(\textit{a}, \textit{c}) \in \textit{S} \circ \textit{R}$ e $(\textit{c}, \textit{d}) \in \textit{T}$. Como $(\textit{a}, \textit{c}) \in \textit{S} \circ \textit{R}$, nós podemos de novo usar a definição de composição e escolher algum $\textit{b} \in \textit{B}$ tal que $(\textit{a}, \textit{b}) \in \textit{R}$ e $(\textit{b}, \textit{c}) \in \textit{S}$. Agora que temos $(\textit{b}, \textit{c}) \in \textit{S}$ e $(\textit{c}, \textit{d}) \in \textit{T}$, nós podemos concluir que $(\textit{b}, \textit{d}) \in \textit{T} \circ \textit{S}$. Semelhante, desde que $(\textit{a}, \textit{b}) \in \textit{R}$ e $(\textit{b}, \textit{d}) \in \textit{T} \circ \textit{S}$,  segue que $(\textit{a}, \textit{d}) \in (\textit{T} \circ \textit{S}) \circ \textit{R}$. 
\\
\\
\begin{center}
\textbf{Exercícios}
\end{center}

1. Encontre os domínios e imagens das relações seguintes.
\\

(a) \{$(\textit{p}, \textit{q}) \in \textit{P} \times \textit{P}$ | a pessoa \textit{p} é pai da pessoa \textit{q}\}, onde \textit{P} é o conjunto de todas as pessoas vivas.
\\

(b) \{$(\textit{x}, \textit{y}) \in \mathbb{R}^{2} | \textit{y} > \textit{x}^{2}$ \}.
\\
2. Encontre os domínios e as imagens das relações seguintes.
\\

(a) \{($\textit{p}, \textit{q}) \in \textit{P} \times \textit{P}$ | a pessoa\textit{p} é irmão da pessoa \textit{q}\}, onde \textit{P} é o conjunto de todas as pessoas vivas.
\\

(b) \{$(\textit{x}, \textit{y}) \in \mathbb{R}^{2}$ | $\textit{y}^{2} = 1 - 2/(\textit{x}^{2} + 1)$\}.
\\
3. Seja \textit{L} e \textit{E} relações definidas nas partes 4 e 5 do exemplo 4.2.2. Descreva as relações seguintes:
\\

(a) $\textit{L}^{-1} \circ \textit{L}$.
\\

(b) $\textit{E} \circ (\textit{L}^{-1} \circ \textit{L})$.
\\
4. Suponha que \textit{A} = \{1, 2, 3\}, \textit{B} = \{4, 5, 6\}, \textit{R} = \{(1, 4), (1, 5), (2, 5), (3, 6)\}, e \textit{S} = \{(4, 5), (4, 6), (5, 4), (6, 6)\}. Note que \textit{R} é uma relação de \textit{A}para \textit{B} e \textit{S} é uma relação de \textit{B} para \textit{B}. Encontre as relações seguintes:
\\

(a) $\textit{S} \circ \textit{R}$.
\\

(b) $\textit{S} \circ \textit{S}^{-1}$.
\\
\\
5. Suponha que \textit{A} = \{1, 2, 3\}, \textit{B} = \{4, 5\}, \textit{C} = \{6, 7, 8\}, \textit{R} = \{(1, 7), (3, 6), (3, 7)\}, e \textit{S} = \{(4, 7), (4, 8), (5, 6)\}. Note que \textit{R} é uma relação de \textit{A} para \textit{C} e \textit{S} é uma relação de \textit{B} para \textit{C}. Encontre as seguintes relações:
\\

(a) $\textit{S}^{-1} \circ \textit{R}$.
\\

(b) $\textit{R}^{-1} \circ \textit{S}$.
\\
\\
6. (a) Prove a parte 3 do teorema 4.2.5 imitando a prova da parte 2 do texto.
\\

(b) De uma alternativa para a prova da parte 3 do teorema 4.2.5 mostrando seguimento das partes 1 e 2.
\\

(c) Complete a prova da parte 4 do teorema 4.2.5.
\\

(d) Prove a parte 5 do teorema 4.2.5.
\\
\\
7. Seja \textit{E} = \{$(\textit{p}, \textit{q}) \in \textit{P} \times \textit{P}$ | a pessoa \textit{p} é um inimigo da pessoa \textit{q}\}, e \textit{F} = \{$(\textit{p}, \textit{q}) \in \textit{P} \times \textit{P}$ | a pessoa \textit{p} é amiga da pessoa \textit{q}\}, onde \textit{P} é o conjunto de todas as pessoas. O que o ditado "o inimigo do meu inimigo é meu amigo" quer dizer sobre a relação \textit{E} e \textit{F}? 
\\
8. Suponha \textit{R} é uma relação de \textit{A} para \textit{B} e \textit{S} é uma relação de \textit{B} para \textit{C}.
\\

(a) Prove que $Dom(\textit{S} \circ \textit{R}) \subseteq Dom(\textit{R})$.
\\

(b) Prove que se $Ran(\textit{R}) \subseteq Dom(\textit{S})$ então $Dom(\textit{S} \circ \textit{R}) = Dom(\textit{R})$.
\\

(c) Formule e prove teoremas similar sobre Ran($\textit{S} \circ \textit{R}$).
\\
9. Suponha \textit{R} e \textit{S} são relações de \textit{A} para \textit{B}. As supostas declarações são verdadeiras ? Justifique sua resposta com provas ou contra exemplos.
\\
 
(a) $\textit{R} \subseteq Dom(\textit{R}) \times Ran(\textit{R})$.
\\

(b) Se $\textit{R} \subseteq \textit{S}$ então $\textit{R}^{-1} \subseteq \textit{S}^{-1}$. 
\\

(c) $(\textit{R} \cup \textit{S})^{-1}$ = $\textit{R}^{-1} \cup \textit{S}^{-1}$.
\\
10. Suponha \textit{R} é uma relação de \textit{A} para \textit{B} e \textit{S} é uma relação de \textit{B} para \textit{C}. Prove que $\textit{S} \circ \textit{R} = \emptyset$ sse Ran(\textit{R}) e Dom(\textit{S}) são disjuntos.
\\
11. Suponha \textit{R} é uma relação  de \textit{A} para \textit{B} e \textit{S} e \textit{T} são relações de \textit{B} para \textit{C}.
\\

(a) Prove que $(\textit{S} \circ \textit{R})$\textbackslash$(\textit{T} \circ \textit{R}) \subseteq (\textit{S}$\textbackslash $\textit{T}) \circ \textit{R}$.
\\

(b) O que está errado com a prova que ($\textit{S}$\textbackslash$\textit{T}) \circ \textit{R} \subseteq (\textit{S} \circ \textit{R})$\textbackslash$(\textit{T} \circ \textit{R})$? 
\\
Prova. Suponha $(\textit{a}, \textit{c}) \in (\textit{S}$\textbackslash$\textit{T}) \circ \textit{R}$. Então nós podemos escolher algum $\textit{b} \in \textit{B}$ tal que $(\textit{a}, \textit{b}) \in \textit{R}$ e $(\textit{b}, \textit{c}) \in \textit{S}$\textbackslash\textit{T}), então (\textit{b}, \textit{c}) $\in \textit{S}$ e $(\textit{b}, \textit{c}) \not\in \textit{T}$. Como $(\textit{a}, \textit{b}) \in \textit{R}$ e $(\textit{b}, \textit{c}) \in \textit{S}$,  $(\textit{a}, \textit{b}) \in \textit{S} \circ \textit{R}$. E $(\textit{a}, \textit{b}) \in \textit{R} e (\textit{b}, \textit{c}) \in \textit{T}$, $(\textit{a}, \textit{c}) \not\in \textit{T} \circ \textit{R}$. Então $(\textit{a}, \textit{c}) \in (\textit{S} \circ \textit{R})$\textbackslash$(\textit{T} \circ \textit{R})$. Como (\textit{a}, \textit{c}) era arbitrário, isso mostra que (\textit{S}\textbackslash\textit{T}) $\circ \textit{R} \subseteq (\textit{S} \circ \textit{R})$\textbackslash$(\textit{T} \circ \textit{R})$.
\\

(c) É verdadeiro que (\textit{S}\textbackslash
\textit{T}) $\circ \textit{R} \subseteq (\textit{S} \circ \textit{R})$\textbackslash$(\textit{T} \circ \textit{R})$? Justifique sua resposta com uma prova ou um contra exemplo.
\\
\\
12. Suponha \textit{R} uma relação de \textit{A} para \textit{B} e \textit{S} e \textit{T} relações de \textit{B} para \textit{C}. As seguintes declarações são verdadeiras? Justifique sua resposta com provas ou contra exemplos.
\\

(a) Se $\textit{S} \subseteq \textit{T}$ então $\textit{S} \circ \textit{R} \subseteq \textit{T} \circ \textit{R}$.
\\

(b) $(\textit{S} \cap \textit{T}) \circ \textit{R} \subseteq (\textit{S} \circ \textit{R} \cap (\textit{T} \circ \textit{R})$.
\\

(c) $(\textit{S} \cap \textit{T}) \circ \textit{R}$ = $(\textit{S} \circ \textit{R} \cap (\textit{T} \circ \textit{R})$.
\\

(d) $(\textit{S} \cup \textit{T}) \circ \textit{R} \subseteq (\textit{S} \circ \textit{R} \cup (\textit{T} \circ \textit{R})$.

